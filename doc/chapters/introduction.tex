%
% introduction.tex
%
% Copyright © 2021 by Universidade Federal de Santa Catarina.
%
% Documentation of Interstage Interface Panels
%
% This work is licensed under the Creative Commons Attribution-ShareAlike 4.0
% International License. To view a copy of this license,
% visit http://creativecommons.org/licenses/by-sa/4.0/.
%

%
% \brief Introduction file.
%
% \author Yan Castro de Azeredo <yan.azeredo@spacelab.ufsc.br>
%
% \institution Universidade Federal de Santa Catarina (UFSC)
%
% \version 2.0.0
%
% \date 05/07/21 (DD/MM/YY)
%
% \note Credits to Gabriel Mariano Marcelino <gabriel.marcelino@spacelab.ufsc.br> for the creation of the SpaceLab Template
%

\chapter{Introduction} \label{ch:introduction}

The Interstage Interface Panels (IIP) are four vertical internally mounted Printed Circuit Boards (PCB\nomenclature{\textbf{PCB}}{\textit{Printed Circuit Board.}}) designed 
to give external access up to four modules inside of a 2U or 3U CubeSat during 
final assembly, integration and testing (AIT\nomenclature{\textbf{AIT}}{\textit{Assembly, integration and testing.}}) before launch.
The complete set of the boards allow the nanosatellite to be charged, programed and debugged. 
The usage of this project is taking into account the use of a MSP-FET: MSP430 Flash Emulation Tool from Texas Instruments for JTAG programing and debugging, UART debugging through a mini USB type B port interfacing the FT4232H USB bridge Integrated Circuit (IC\nomenclature{\textbf{IC}}{\textit{Integrated Circuit.}}) from FTDI, a JST XH header for charging internal batteries and a Remove Before Flight (RBF\nomenclature{\textbf{RBF}}{\textit{Remove Before Flight.}}) pin header.
One board variant features an opening for a M12 camera lens.  
These tools and methodology for testing are definied directly from the project main use on the FloripaSat-2 mission \cite{floripasat2} been done by SpaceLab UFSC.

\begin{figure}[!ht]
    \centering
    \includegraphics[width=\textwidth]{figures/iip_fullset.png}
    \caption{Interstage Interface Panels boards.}
    \label{iip-fullset}
  \end{figure}

All the project, source and documentation files are available freely on a GitHub repository \cite{interface-board-repo}.