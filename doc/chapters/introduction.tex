%
% introduction.tex
%
% Copyright © 2020 by Universidade Federal de Santa Catarina.
%
% Interstage Interface Panels (IIP)
%
% This work is licensed under the Creative Commons Attribution-ShareAlike 4.0
% International License. To view a copy of this license,
% visit http://creativecommons.org/licenses/by-sa/4.0/.
%

%
% \brief: Introduction file.
%
% \author: Yan Castro de Azeredo <yan.azeredo@spacelab.ufsc.br>
%
% \institution: Universidade Federal de Santa Catarina (UFSC)
%
% \version: 0.1.1
%
% \date: 14/07/20 (DD/MM/YY)
%
% Credits to Gabriel Mariano Marcelino <gabriel.marcelino@spacelab.ufsc.br> for the creation of the SpaceLab Template
%

\chapter{Introduction} \label{ch:introduction}

The Interstage Interface Panels (IIP) are three vertical mounted PCBs designed 
to give external access up to four modules inside of a 2U CubeSat during 
final assembly, integration and testing (AIT) before launch.
The complete set of the boards allow the nanosatellite to be charged, programed and debugged. The usage of this hardware platform is taking into account the use of a MSP-FET: MSP430 Flash Emulation Tool from Texas Instruments for JTAG programing and debugging, UART debugging through a mini USB B port interfacing the FT4232H USB bridge IC from FTDI, a JST XH connector for charging internal batteries and a Remove Before Flight (RBF) pin header. These tools and methodology for testing are definied directly from the project main use on the GOLDS-UFSC mission \cite{golds-ufsc} been done with the support of SpaceLab UFSC.

\begin{figure}[!ht]
    \centering
    \includegraphics[width=\textwidth]{figures/iip_fullset.png}
    \caption{Interstage Interface Panels fullset}
    \label{iip-fullset}
  \end{figure}

All the project, source and documentation files are available freely on a GitHub repository \cite{interface-board-repo}.