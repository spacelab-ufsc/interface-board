%
% system_overview.tex
%
% Copyright © 2020 by Universidade Federal de Santa Catarina.
%
% Interstage Interface Panels (IIP)
%
% This work is licensed under the Creative Commons Attribution-ShareAlike 4.0
% International License. To view a copy of this license,
% visit http://creativecommons.org/licenses/by-sa/4.0/.
%

%
% \brief: System overview chapter.
%
% \author: Yan Castro de Azeredo <yan.azeredo@spacelab.ufsc.br>
%
% \institution: Universidade Federal de Santa Catarina (UFSC)
%
% \version: 0.1.1
%
% \date: 14/07/20 (DD/MM/YY)
%
% Credits to Gabriel Mariano Marcelino <gabriel.marcelino@spacelab.ufsc.br> for the creation of the SpaceLab Template
%

\chapter{System Overview} \label{ch:system-overview}

\section{Block diagram}

On figure \ref{fig:block-diagram} is displayed the full system block diagram with external devices such as the MSP-FET and a personal computer (PC) during normal usage of IIP. Up to four cubesat modules can be acessed from its interfaces, been though the pin headers or the USB port. For a specific project, the core modules (EPS, OBDH and TTC) and a optional payload are already represented with their respective interconections. These connections are done internaly with PicoBlade connectors, which are compatible with SpaceLab's modules. 

\begin{figure}[!ht]
    \begin{center}
        \includegraphics[width=\textwidth]{figures/iip_block_diagram.png}
        \caption{IIP hardware block diagram.}
        \label{fig:block-diagram}
    \end{center}
\end{figure}

\section{Board numeration} %\Label{ssec:board-numeration}

Since the IIP is divided up to 3 boards, a numeration was adopted for better refering each PCB on this document. The numeration followed the criteria of the mounting sides determined by the carthesian coordinates of a Poly Picosatellite Orbital Deployer (P-POD) \cite{p-pod-user-guide}. Starting the numeration "Nº1" from the refereced -X plane, the other boards are classified clock-wise that can be found showed on figure \ref{fig:iip-numeration}, each labeled axis can be better seen on figure \ref{fig:iip-all-boards-top-bottom}. The boards also received names with their unique functionality, Nº1 board is also called "IIP Charge", Nº2 board the "IIP RBF" and Nº3 board the "IIP Quad UART", these nominations are present the PCBs source files. Details about the dimensions and mouting on a 2U structure can be read in the assembly chapter \ref{ch:board-assembly}.

\begin{figure}[!ht]
  \centering
    \includegraphics[width=0.5\linewidth]{figures/iip_n1_perspective.png}
    \caption{IIP Nº1 Board}
    \includegraphics[width=0.5\linewidth]{figures/iip_n2_perspective.png}
    \caption{IIP Nº2 Board}
    \includegraphics[width=0.5\linewidth]{figures/iip_n3_perspective.png}
    \caption{IIP Nº3 Board}
    \label{fig:iip-numeration}
\end{figure}