%
% introduction.tex
%
% Copyright © 2020 by Universidade Federal de Santa Catarina.
%
% Interstage Interface Panels of FloripaSat-2
%
% This work is licensed under the Creative Commons Attribution-ShareAlike 4.0
% International License. To view a copy of this license,
% visit http://creativecommons.org/licenses/by-sa/4.0/.
%

%
% \brief: Introduction file.
%
% \author: Yan Castro de Azeredo <yan.ufsceel@gmail.com>
%
% \institution: Universidade Federal de Santa Catarina (UFSC)
%
% \version: 0.1.0
%
% \date: 14/07/20 (DD/MM/YY)
%
%  Credits to Gabriel Mariano Marcelino <gabriel.mm8@gmail.com> for the creation of the SpaceLab Template
%

\chapter{Introduction} \label{ch:introduction}

The Interstage Interface Panels (IIP) are vertical mounted PCBs designed 
to give external access to the modules inside of a 2U CubeSat during 
final assembly, integration and testing (AIT) before launch.
IIP is composed of 3 different boards, the complete set allows for the nanosatellite
to be charged, programed and debugged. The project is inspired by other solutions already developed, 
such as the GOMspace NanoUtil Interstage. The design was developed taking into account the use of 
a MSP-FET: MSP430 Flash Emulation Tool from Texas Instruments for JTAG interfacing,
and the FT4232H USB bridge IC from Future Technologies Devices International (FTDI) for quad UART debug channels. 

The main motivation for the project was the necessity of a custom off the shelf
solution for making the external interface of the FloripaSat-2 CubeSat. The Interface Board of the 
first CubeSat launched by SpaceLab UFSC, FloripaSat-1, was designed for 1U structure. While both interfaces
are very different, some hardware choices were inheritated because of the already features in 
the core modules used, such as the use of PicoBlade connectors for internal connection
betweem modules and interface.