%
% system_overview.tex
%
% Copyright © 2020 by Universidade Federal de Santa Catarina.
%
% Interstage Interface Panels of FloripaSat-2
%
% This work is licensed under the Creative Commons Attribution-ShareAlike 4.0
% International License. To view a copy of this license,
% visit http://creativecommons.org/licenses/by-sa/4.0/.
%

%
% \brief: System overview chapter.
%
% \author: Yan Castro de Azeredo <yan.ufsceel@gmail.com>
%
% \institution: Universidade Federal de Santa Catarina (UFSC)
%
% \version: 0.1.0
%
% \date: 14/07/20 (DD/MM/YY)
%
%  Credits to Gabriel Mariano Marcelino <gabriel.mm8@gmail.com> for the creation of the SpaceLab Template
%

\chapter{System Overview} \label{ch:system_overview}

\section{Block diagram}

\begin{figure}[!ht]
    \begin{center}
        \includegraphics[width=0.8\textwidth]{figures/IIP_SemiUSB_BlockDiagram.png}
        \label{fig:block-diagram}
        \caption{IIP Semi USB full system block diagram.}
    \end{center}
\end{figure}

\section{Board numeration}

Since the IIP is divided up to 3 boards, a numeration was designed for better refering each PCB.
The numeration followed the criteria of the mounting sides determined by the carthesian coordinates
followed by the CubeSat's modules and Fit Check on the P-POD. Starting the numeration N1 from the
refereced -X plane front of the nanosatellite the other boards are classified clock-wise.


\subsection{Semi USB}

Image A
(full board assembly image) 

\begin{figure}[!ht]
    \begin{center}
        \includegraphics[width=0.7\textwidth]{figures/IIP_SemiUSB_N1_charge_top_print.png}
        \label{fig:block-diagram}
        \caption{Nº1 board top view.}
    \end{center}
\end{figure}

\begin{figure}[!ht]
    \begin{center}
        \includegraphics[width=0.7\textwidth]{figures/IIP_SemiUSB_N1_charge_bottom_print.png}
        \label{fig:block-diagram}
        \caption{Nº1 board bottom view.}
    \end{center}
\end{figure}

\begin{figure}[!ht]
    \begin{center}
        \includegraphics[width=0.7\textwidth]{figures/IIP_SemiUSB_N2_RBF_top_print.png}
        \label{fig:block-diagram}
        \caption{Nº2 board top view.}
    \end{center}
\end{figure}

\begin{figure}[!ht]
    \begin{center}
        \includegraphics[width=0.7\textwidth]{figures/IIP_SemiUSB_N2_RBF_bottom_print.png}
        \label{fig:block-diagram}
        \caption{Nº2 board bottom view.}
    \end{center}
\end{figure}

% Image D
% N3: Board mounted and directed on the +Y plane


\section{Mechanical dimensions}

The dimensions for the mouting holes on the X and Y planes of the ISIS 2 CubeSat structure
are different, for this reason IIP has two mechanical dimension standards.

Image A
(Mechanical dimensions for N1 and N3 boards mounted in the Y plane)

Image B
(Mechanical dimensions for N2 board)